%%%%%%%%%%%%%%%%%%%%%%%%%%%%%%%%%%%%%%%%%%%%%%%%%%%
%
%  New template code for TAMU Theses and Dissertations starting Spring 2021.
%
%
%  Last Updated: 1/13/2021
%
%%%%%%%%%%%%%%%%%%%%%%%%%%%%%%%%%%%%%%%%%%%%%%%%%%%

% THIS TEMPLATE IS THE MOST
% CURRENT. SEE THE FILES README.TXT AND NEWCHANGES.TXT
% FOR MORE INFORMATION.

\documentclass[12pt]{report}

\usepackage{tamuconfig}


% Most of the packages that set the default settings
% for the document have moved to the style file
% tamuconfig.sty. This includes

%These next lines change the font. Fixes for certain
%fonts will be implemented in a future release.

%Comment this line if you do not wish to use Times
%New Roman. The font used will then be the LaTeX
%default of Computer Modern.
\usepackage{times}
%\usepackage{cmbright}
\usepackage[T1]{fontenc}

% For natbib-style references, uncomment this.
%\usepackage{natbib}

%This package allows for the use of graphics in the
%document.
\usepackage{graphicx}

%If you have JPEG format images, add .jpg as an
%allowed file extension below. Same for Bitmaps (.bmp).
\DeclareGraphicsExtensions{.png}

%It is best practice to keep all your pictures in
%one folder inside the main directory in which your
%TeX file is kept. Here the folder is named "graphic."
%Replace the name here with your folder's name, if needed.
%The period is needed due to relative referencing.
\graphicspath{ {./graphic/} }

% For quick document navigation.
\usepackage[hidelinks]{hyperref}

%%%%%%%%%%%%%%%%%%%%%%%%%%%%%%%%%%%%%%%%%%%%%%%%%%%%%%%%%
%Please place all your personal packages here. Check to
%see if the packages you wish to use are not already
%declared above. Placing all your personal packages
%here allows me to determine if there are any package
%issues in compilation, as well as any conflicts
%that may arise by the order of loading.
%
%%%%%%%%%%%%%%%%%%%%%%%%%%%%%%%%%%%%%%%%%%%%%%%%%%%%%%%%%
%%%%%%%%%%%%%%%%%%%%%%%%%%%%%%%%%%%%%%%%%%%%%%%%%%%%%%%%%
%Begin student defined packages.
%%%%%%%%%%%%%%%%%%%%%%%%%%%%%%%%%%%%%%%%%%%%%%%%%%%%%%%%%

\usepackage[printonlyused]{acronym}
\usepackage{etoolbox}
\makeatletter
\patchcmd{\chapter}{\if@openright\cleardoublepage\else\clearpage\fi}{}{}{}
\expandafter\patchcmd\csname AC@\AC@prefix{}@acro\endcsname{{#3}}{{\MakeUppercase{#3}}}{}{}
\makeatother

%%%%%%%%%%%%%%%%%%%%%%%%%%%%%%%%%%%%%%%%%%%%%%%%%%%%%%%%%
%End student defined packages.
%%%%%%%%%%%%%%%%%%%%%%%%%%%%%%%%%%%%%%%%%%%%%%%%%%%%%%%%%

% End preamble. Document begins below.

\begin{document}

%The title of your document goes here.
%Spacing may need to be adjusted if your title is long
%and pushes the copyright off the page.
\renewcommand{\tamumanuscripttitle}{ The Title of Your Thesis or Dissertation Goes In This Space To Let Us Know What Your Document is About}


%Type only Thesis, Dissertation, or Record of Study.
\renewcommand{\tamupapertype}{Thesis}

%Your full name goes here, as it is in university records. Check your student record on Howdy if there is any mismatch.
\renewcommand{\tamufullname}{Aggie M. Student}

%The degree title goes here. See the GPS site for more info.
\renewcommand{\tamudegree}{Master of Science}
\renewcommand{\tamuchairone}{Chair Name}


% Uncomment out the next line if you have co-chairs.  You will also need to edit the titlepage.tex file.
%\newcommand{\tamuchairtwo}{Additional Chair Name}
\renewcommand{\tamumemberone}{Committee Member 1}
\renewcommand{\tamumembertwo}{Committee Member 2}
\renewcommand{\tamumemberthree}{Committee Member 3}
\renewcommand{\tamudepthead}{Head of Department}

%Type only May, August, or December.
\renewcommand{\tamugradmonth}{December}
\renewcommand{\tamugradyear}{2021}
%Your department name goes here.
\renewcommand{\tamudepartment}{Mathematics}

% Boilerplate
%%%%%%%%%%%%%%%%%%%%%%%%%%%%%%%%%%%%%%%%%%%%%%%%%%%%%%%%%%%%%%%%%%%%%%%%%%%%%%%%
% TITLE PAGE
%%%%%%%%%%%%%%%%%%%%%%%%%%%%%%%%%%%%%%%%%%%%%%%%%%%%%%%%%%%%%%%%%%%%%%%%%%%%%%%%

\providecommand{\tabularnewline}{\\}

\begin{titlepage}
    \begin{center}
        \MakeUppercase{\tamumanuscripttitle}
        \vspace{4em}

        A \tamupapertype

        by

        \MakeUppercase{\tamufullname}

        \vspace{4em}

        \begin{singlespace}
            Submitted to the Office of Graduate and Professional Studies of \\

            Texas A\&M University \\

            in partial fulfillment of the requirements for the degree of \\
        \end{singlespace}

        \MakeUppercase{\tamudegree}
        \par\end{center}
    \vspace{2em}

    \begin{tabular}{ll}
        %    & \tabularnewline
        %    & \cr
        Chair of Committee & \tamuchairone\tabularnewline
        Committee Members  & \tamumemberone\tabularnewline
                           & \tamumembertwo\tabularnewline
                           & \tamumemberthree\tabularnewline
        Department Head    & \tamudepthead\tabularnewline
    \end{tabular}

    \vspace{3em}

    \begin{center}

        \tamugradmonth \hspace{2pt} \tamugradyear

        \vspace{3em}

        Major Subject: \tamudepartment \par

        \vspace{3em}

        Copyright \tamugradyear \hspace{.5em} \tamufullname

        \par\end{center}

\end{titlepage}
\pagebreak{}

\include{boilerplate/B2_Abstract}
\include{boilerplate/B3_Dedication}
\include{boilerplate/B4_Acknowledgements}
\include{boilerplate/B5_Contributors}
%%%%%%%%%%%%%%%%%%%%%%%%%%%%%%%%%%%%%%%%%%%%%%%%%%%%%%%%%%%%%%%%%%%%%%%%%%%%%%%%
% NOMENCLATURE
%%%%%%%%%%%%%%%%%%%%%%%%%%%%%%%%%%%%%%%%%%%%%%%%%%%%%%%%%%%%%%%%%%%%%%%%%%%%%%%%

\chapter*{NOMENCLATURE}
\addcontentsline{toc}{chapter}{NOMENCLATURE}  % Needs to be set to part, so the TOC doesn't add 'CHAPTER ' prefix in the TOC.

%A note about aligning: These entries will align
%themselves according to the ampersand (&).
%No extra spaces are needed, as seen in some of
%the entries below.

\begin{acronym}
    \acro{GPS}{graduate and professional school}
\end{acronym}


\pagebreak{}
\include{boilerplate/B7_TableOfContents}

% Chapters
\include{chapters/C1_Introduction}
\include{chapters/C2_Figures}
\include{chapters/C3_Tables}
\include{chapters/C4_Conclusions}

%The next line is the format for inserting new sections.
%Replace the name "newsection"  with the name of your
%new section file.
%\include{data/newsection}

%fix spacing in bibliography, if any...
%%%%%%%%%%%%%%%%%%%%%%%%%%%%%%%%%%%%%%%%%%%%%%%%%%%%%%%%%%%%%
\let\oldbibitem\bibitem
\renewcommand{\bibitem}{\setlength{\itemsep}{0pt}\oldbibitem}
%%%%%%%%%%%%%%%%%%%%%%%%%%%%%%%%%%%%%%%%%%%%%%%%%%%%%%%%%%%%%%%
%The bibliography style declared is the IEEE format. If
%you require a different style, see the document
%bibstyles.pdf included in this package. This file,
%hosted by the University of Vienna, shows several
%bibliography styles and examples of in-text citation
%and a references page.
\bibliographystyle{ieeetr}

\phantomsection
\addcontentsline{toc}{chapter}{REFERENCES}

\renewcommand{\bibname}{{\normalsize\rm REFERENCES}}

%This file is a .bib database that contains the sources.
%This removes the dependency on the previous file
%bibliography.tex.
\bibliography{references}

%This next line includes appendices. The file
%appendix.tex contains commands pointing to
%the appendix files; be sure to change these
%pointers if you end up changing the filenames.
%Leave this commented if you will not need
%appendix material.

\begin{appendices}
    \titleformat{\chapter}{\centering\normalsize}{APPENDIX \thechapter}{0em}{\vskip .5\baselineskip\centering}
    \renewcommand{\appendixname}{APPENDIX}

    \input{appendices/A1_First}
    \input{appendices/A2_Second}

\end{appendices}


\end{document}
